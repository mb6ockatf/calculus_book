\documentclass[a5paper,10pt]{book}
\usepackage{graphicx,amssymb,amsmath,lmodern,wrapfig,csquotes,savetrees,
	hyperref,multicol,bookmark,amsthm,thmtools,mathtools,array,
	tkz-euclide,gensymb,twemojis,lettrine}
% \usepackage[lighttt]{lmodern}
% \usepackage[fontsize=10pt]{fontsize}
% \usepackage[parfill]{parskip}
\usepackage[T2A]{fontenc}
\usepackage[utf8]{inputenc}
\usepackage[russian,english]{babel}
\DeclareGraphicsRule{.1}{mps}{*}{}
\DeclareGraphicsRule{.2}{mps}{*}{}

\renewcommand{\chaptername}{Глава}
\addto\captionsenglish{\renewcommand{\chaptername}{Глава}}
\addto\captionsenglish{\renewcommand{\contentsname}{Содержание}}

\theoremstyle{definition}
\newtheorem{definition}{Определение}
\theoremstyle{theorem}
\newtheorem{theorem}{Теорема}
\renewcommand{\proofname}{Доказательство}
\newtheorem*{remark}{Remark}
\newtheorem*{example}{Пример}
% \newtheorem{definition}{Определение}[section]
% \setlength{\parindent}{0pt}
% \usepackage[a5paper]{geometry}
% \geometry{a4paper, left=6mm, right=6mm, bottom=15mm, top=6mm}
%\geometry{a4paper, margin=0in, bottom=15mm, top=6mm}
\author{А. Б. Бакушинский, В. К. Власов}

\title{Элементы высшей математики и численных методов}
\date{1968}
\begin{document}
\maketitle
\tableofcontents
\section{Предисловие}
В этой книге сделана попытка отобрать и доступно изложиь те разделы математики
(в том числе и вычислительной), которые не входят в обычные программы
общеобразовательных средних школ, но которые, по мнению авторов, необходимо
знать лаборанту-программисту. Поэтому книгу можно рассматривать как учебное
пособие для учащихся школ и различных курсов, готовящих программистов. Кроме
того, лица со средним образованием могут её использовать для самостоятельного
ознакомления с элементами высшей математики и методов вычислений.

Особенностью книги является тесное переплетение вопросов вычислительной
математики и математического анализа. Разделы вычислительной математики
помещены обычно после необходимых для их изучения разделов анализа.

В конце параграфов приведены иллюстрирующие материал упражнения, которых,
однако, недостаточно для глубокого усвоения курса. Большое количество
подходящих упражнений можно найти, например, в следующих задачниках:

\begin{itemize}
	\item \textit{В. П. Минорский.} Сборник задач по высшей математике. М.,
		<<Наука>>, 1964.
	\item \textit{Г. Н. Берман.} Сборник задач по курсу математического анализа. М.,
		<<Наука>>, 1965.
\end{itemize}

Последовательное изучение предлагаемого курса можно начинать с 9-го класса при
условии параллельного прохождения обычной программы по математике для средних
школ.

Авторы приносят глубокую благодарность доценту В. М. Алексееву, преподавателю
физико-математической школы-интерната А. А. Шершевскому и доценту Н. П.
Жидкову, прочитавшим книгу в рукописи и сделавшим ряд очень полезных замечаний.

Особую признательность авторы выражают профессору И. С. Березину, советы
которого по содержанию и методике изложения существенно способствовали
улучшению качества книги.

\begin{flushright}
	\textit{А. Бакушинский\\
		В. Власов
	}
\end{flushright}
\chapter{Элементарная теория погрешностей}
\section{Множества. Вещественные числа}

\lettrine[lines=3]{О}{}
дним из наиболее важных понятий в современной математике является понятие
множества. Это понятие настолько общее, что ему нельзя дать какого-либо
определения. Следует только иметь ввиду, что слово <<множество>> эквивалентно
словам: <<совокупность>>, <<собрание элементов>>, <<семейство>>, <<класс
элементов>> и т. п. Можно говорить, например, о множестве целых положительных
чисел, множестве людей в комнате, множестве всех видимых звёзд и т. д.

Множество может содержать как конченое число элементов (множество людей в
комнате), так и бесконеченое их число (множество всех целых чисел).

Множество, сожержащее конечное число элементов, называется \textit{конечным}
множеством. Множество, состоящее из бесконечного числа элементов, называется
\textit{бесконечным} множеством.

Большой интерес для нас будут иметь множества, элементы которых -- вещественные
(или действительные) числа.

Напомним некоторые определения и факты, относящиеся к понятию вещественного
числа, нужные нам в дальнейшем. Как извество, \textit{вещественным} числом
называется любое целое, рациональное или иррациональное число.
\textit{Рациональные} числа -- это числа вида $\frac{p}{q}$, где $p$ и $q$
целые. Они могут быть представлены в виде конечной или бесконечной периодической
десятичной дроби. Однако одних рациональных чисел недостаточно для решения даже
очень простых алгебраических задач. Например, уравнение $x^2 - 2 = 0$
неразрешимо в множестве рациональных чисел (не существует 2 таких целых чисел
$p$ и $q$, что $(\frac{p}{q})^2 - 2 = 0$).

Назовём \textit{иррациональным} числом всякую непериодическую бесконечную
десятичную дробь.

После введения иррациональных чисел уравнение $x^2 - 2 = 0$ становится
разрешимым (его решения: $\pm\sqrt{2} = \pm 1.414\ldots$). С введением
иррациональных чисел получают своё решение и другие математические задачи, в
частности задача определения длинны отрезка, не соизмеримого с выбранной
единицей масштаба, и т. п.

Очень полезно для дальнейшего представление вещественных чисел в виде точек на
некоторой прямой. Эта прямая называется числовой и строится следующим образом:
на прямой выбирают произвольную точку $O$ и называют её началом отсчёта. Затем
задают на прямой положительное направление и единицу масштаба. Тогда для каждой
точки $M$ на прямой можно измерить расстояние от этой точки до начала отсчёта с
помощью заданного масштаба. Таким образом, каждой точке на прямой можно
поставить в соответствие некоторое вещественное число, и притом только одно,
характеризующее расстояние от этой точки до начала отсчёта. И наоборот, каждому
вещественному числу можно поставить в соответствие некоторую точку на данной
прямой, и притом только одну, расстояние от которой до начала отсчёта
выражается этим числом.

Итак, мы установили соответствие между всеми вещественными числами и всеми
точками числовой прямой. Поэтому очень часто различные вещественные числа
называют просто точками.

\textit{Интервалом} с концами в точках $a$ и $b$ называется множество точек,
удовлетворяющих неравенству $a < x < b$. Сам интервал обозначают $(a, b)$,
принадлежность числа $x$ к этому интервалу обозначают $x \in (a, b)$ ($\in$ --
знак принадлежности). На числовой прямой интервал изображают так, как это
указано на рисунке 1а. Стрелки в точках $a$ и $b$ указывают, что эти точки не
входят в множество, которое мы назвали интервалом.

Интервал называется \textit{полуоткрытым}, если в него входит один из его
концов. Полуоткрытый интервал $(a, b]$ -- множество точек $x$, которые
удовлетворяют неравенству (рис. 1б): $a < x \leq b$.

Полуоткрытый интервал $[a, b]$ -- множество точек $x$, удовлетворяющих
неравенству (рис. 1в): $a \leq x < b$.

И, наконец, \textit{замкнутым интервалом} или \textit{отрeзком} $[a, b]$,
называется множество точек $x$, которые удовлетворяют неравенству (рис. 1г): $a
\leq x \leq b$.

Объединением интервалов называется совокупность точек, принадлежащих хотя бы
одному из интервалов, например, в объединение интервалов $(a_1, b_1)$ и $a_2,
b_2)$ (рис. 1д) входят все точки обоих интервалов.

\includegraphics{fig1.mps}
% TODO: тут должны быть рисунки, стр. 11

\begin{definition}
	Абсолютной величиной (модулем) числа $x$ (обозначается $|x|$)
	называется само число $x$, если $x \geq 0$, и $(-x)$, если $x < 0$:
	$$|x| = \begin{cases}
		x, &\text{если $x \leq 0$,}\\
		-x, &\text{если $x < 0$}
	\end{cases}$$
\end{definition}
Например, $|1| = 1$, а $|-5| = 5$.

Очевидное свойство абсолютных величин:
$$|x| - |y| \leq |x + y| \leq |x| + |y|$$

\section[Погрешности числа]{Источники ошибок. Абсолютная и \linebreak относительная погрешность
числа}
Если решением некоторой задачи является число, то практически далеко не всегда
мы можем получить это число совершенно точно. Причины этого следующие.

\begin{enumerate}
	\item Числа, которые участвуют в операциях, обычно записываются в виде
		десятичных дробей. Если исходные числа были иррациональными, то
		точно записать их в виде десятичных дробей мы, конечно, не
		сможем (так как такая запись содержит бесконечное число цифр).
		Поэтому приходится вместо исходных иррациональных чисел
		оперировать с рациональными числами, полученными их данного
		иррационального числа, если в его записи оставить только первые
		несколько десятичных знаков (например, столько, сколько входит
		в разрядную сетку вычислительной машины).

		Так, вместо числа $\sqrt(2)$ приходится использовать
		рациональные числа $1.14$ или $1.41$ или $1.4142$ и т. д.
		Естественно, что результат действий будет содержать некоторую
		ошибку (погрешность) тем большую, чем меньше десятичных знаков
		содержат рациональные приближения к исходным иррациональным
		числам. Возникают подобные погрешности и тогда, когда исходные
		данные были рациональными числами. При их записи в виде
		десятичной дроби может получиться бесконечная периодическая
		дробь или конечная дробь, число знаков которой настолько
		велико, что имеющиеся в нашем распоряжении вычислительные
		средства не могут их учесть целиком и некоторое число знаков
		приходится отбрасывать.

		Итак, исходные данные, промежуточные результаты и окончательные
		результаты мы обычно не можем записать совершенно точно, а
		округляем их. Это одна из причин неточности ответа.

	\item Очень часто задачи в их непосредственной формулировке либо не
		поддаются решению, либо решение чрезвычайно сложное. Тогда
		заменяют эту задачу другой, более простой, которая даёт
		решение, достаточно близкое к требуемому, и решают эту более
		простую задачу. Естественно, и в этом случае результат
		получится с какой-то погрешностью, даже если все исходные
		данные и промежуточные результаты будут точными. Говорят, что
		возникает погрешность метода.

	\item Наконец, одним из самых важных источников погрешности результата
		является неточность самих исходных данных. Как правило,
		исходные данные для задачи получают из какого-либо физического
		эксперимента. Любой эксперимент связан с измерениями, а
		измерения всегда производятся с той или иной погрешностью.
		Например, если нам нужно определить площадь прямоугольной
		комнаты, мы берём линейку и измеряем длину и ширину комнаты.
		Как бы мы не старались, а на несколько сантиметров ошибёмся,
		хотя бы из-за того, что сама линейка может быть не абсолютно
		точной. Естественно, и площадь комнаты как результат умножения
		длины на ширину получится не совсем точно. Подобных примеров
		можно привести сколь угодно много.
\end{enumerate}

Количественной характеристикой погрешностей величин служит их абсолютная и
относительная погрешность.

Пусть для величины, точное значение которой есть $x$, мы каким-либо образом
получили приближённое значение $x^*$.

\begin{definition}
	Абсолютная величина разности между промежуточным значением $x$ и его
	приближённым значением $x^*$ называется абсолютной погрешностью
	приближённного числа $x^*$ и обозначается $\Delta_{x^*}$, т. е. $x - x^*| =
	\Delta_{x^*}$
\end{definition}

Как правило, точное значение $x$ нам неизвестно, а следовательно, неизвестна и
абсолютная погрешность $\Delta_{x^*}$. Но зато обычно можно определить число,
которое эта абсолютная погрешность заведомо не превосходит (границу абсолютной
погрешности, определяемую самим способом нахождения числа). Так, взвешивая
какой-либо предмет на аптекарских весах, мы не можем определить точного веса
предмета, на гарантируем, что ошибка взвешивания не более, чем $0.01$ грамма,
т. е. абсолютная погрешность веса не будет превышать $0.01$ грамма.

Однако абсолютная погрешность не всегда достаточно полно характеризует
погрешность вычислений. В самом деле, пусть ошибка при измерении длины
радиоволны равна одному метру. Если при этом измерялась длина волны в диапазоне
длинных волн, то точность хорошая; такая же ошибка при измерении длины волны в
диапазоне УКВ (ультракоротких волн) слишком велика. Поэтому вводят ещё одно
важное понятие -- относительную погрешность.

\begin{definition}
	Относительной погрешностью приближённого значения $x^*$ называется
	отношение абсолютной погрешности $\Delta_{x^*}$ к абсолютному значению
	приближённой величины: $\delta_{x^*} = \frac{\Delta_{x^*}}{|x^*|}$
\end{definition}

Нетрудно видеть, что если абсолютная погрешность всегда имеет ту же размерность,
что и сами величины, то относительная погрешность есть величина безразмерная.

Разумеется, говорить об отноительной погрешности можно только в том случае,
когда $X^* \neq 0$. В дальнейшем мы будем там, где это необходимо, предполагать
это условие выполненным и не будем делать специальной оговорки.

В примере с измерением длин радиоволн абсолютная погрешность равна одному метру.
Если после измерения получим, что длина волны в диапазоне длинных волн 100 м, а
в диапазоне УКВ 4 м, то в первом случае относительная погрешность равна
$\frac{1}{1000} = 0.001$, а во втором $\frac{1}{4} = 0.25$.

Мы будем вполне удовлетворены, если наши часы будут убегать в сутки на 10
секунд, но вряд ли будем довольны, если они будут убегать на 10 секунд каждую
минуту. В первом случае относительная погрешность равна $\frac{10}{24 \cdot 60
\cdot 60} = \frac{1}{8640} < 0.00012$, а во втором $\frac{10}{60} = \frac{1}{6}
\cong 0.17$, абсолютная же погрешность и в том, и в другом случае одинакова --
10 секунд.

\subsubsection{Упражнения}
\begin{enumerate}
	\item Записать число $\pi$ лишь с 2 знаками после запятой и оценить
		абсолютную и относительную погрешность полученного приближённого
		значения.
	\item Комнатный термометр даёт отклонения не более, чем 0.5 градуса. С
		его помощью измерили температуру и получили $20\degree$.
		Требуется оценить абсолютную и относительную погрешности
		полученной величины температуры.
\end{enumerate}

\section{Правила округления}
В предыдущем параграфе мы уже говорили о том, что на практике часто приходится
иметь дело с числами, запись которых в виде десятичной дроби требует бесконечно
много знаков, и с числами, число знаков у которых в такой записи конечно, но
может быть очень большим. Любая вычислительная машина имеет лишь вполне
определённое количество разрядов. Поэтому, чтобы ввести такие числа в машину,
нужно их каким-то образом записать так, чтобы количество цифровых знаков не
превышало количнства цифровых разрядов, имеющихся в машине (округлить число).
Очевидно, что это необходимо делать и тогда, когда мы считаем на бумаге без
помощи машин. Обычно округление производят по следующему правилу (иногда,
правда, пользуются и другими правилами). Пусть какое-то число имеет в своей
записи более, чем $k$, цифорвых знаков и мы хотим округлить его, оставив ровно
$k$ знаков. Тогда если $(k + 1)$-я цифра в записи числа меньше или равна 4, то
все цифры, начиная с $(k + 1)$-й, просто отбрасывают.

Например, если в числе 3.14159265358\ldots мы хотим оставить 5 знаков после
запятой, то получим: 3.14159.

Пусть теперь $(k + 1)$-я цифра больше, чем 5. Тогда мы тоже отбрасываем цифры,
начиная с $(k + 1)$-й, но в оставшемся числе $k$-ю цифру увеличиваем на 1.
Так, если в вышеописанном числе мы хотим оставить шесть цифр после запятой, то
получим: 3.141593.

Если $(k + 1)$-я цифра есть 5, а за ней найдётся хоть одна отличная от 0 цифра,
то поступают, как в предыдущем случае, т. е. отбрасывают все цифры, начиная с $k
+ 1)$-й, а $k$-ю увеличивают на 1.

В нашем примере, оставив 4 цифры после запятой, получим: 3.1416.

Наконец, последний случай, когда $(k + 1)$-я цифра есть 5, а все последующие за
ней цифры -- 0. Тогда поступают так: отбрасывают <<хвост>>, начиная с
$(k + 1)$-й цифры, и если $k$-я цифра чётная, то оставляют её без изменения,
если же $k$-я цифра нечётная, то увеличивают её на 1.

Например, в числе 5.3865 оставим 3 знака после запятой: 5.386; если же округлим
число 7.4235, то получим 7.4234

Всё, что было сказано выше, можно сформулировать так: для того чтобы округлить
число до $k$ десятичных знаков, нужно отбросить все знаки, начиная с $(k +
1)$-го; если при этом отброшенная часть меньше половины единицы $k$-го разряда,
то к $k$-му разряду прибавляют 1; наконец, если отброшенная часть в точности
равна половине единицы $k$-го разряда и $k$-я цифра чётная, то оставляют эту
цифру без изменения, если же нечётная, то увеличивают её на 1.

Легко проверить, что абсолютная погрешность числа, полученного округлением по
этому правилу, не превосходит 5 единиц 1-го отброшенного разряда.

\section{Действия над приближёнными числами}
Производя различные арифметические операции над приближёнными числами, мы
получаем и приближённый ответ. Возникает вопрос: какова погрешность результата,
если известны погрешности исходных данных?
\subsection{Погрешность суммы}
\begin{theorem}[Абсолютная погрешность суммы]
	Абсолютная погрешность суммы приближённых величин не превосходит суммы
	абсолютных погрешностей этих величин.
	\label{thm:abs_max_inaccuracy}
\end{theorem}
\begin{proof}[Доказательство]
	Пусть $x = x_1 + x_2 + \cdots + x_n$ (величины $x_i$ могут быть любого
	знака). Сумма приближённых значений даёт: $x^* = x^*_1 + x^*_2 + \cdots
	+ x^*_n$.
	Вычитая из точного значения суммы приближённое её значение, получим:
	$$x - x^* = (x_1 - x^*_1) + (x_2 - x^*_2) + \cdots + (x_n - x^*_n).$$
	Отсюда
	$$|x - x^*| \leq |x_1 - x^*_1| + |x_2 - x^*_2| + \cdots + |x_n - x^*_n|.$$
	(Это свойство абсолютных величин легко доказывается с помощью метода
	математической индукции (см. Приложение)). Следовательно,
	\begin{equation}
		\Delta_{x^*} \leq \Delta_{x^*_1} + \Delta_{x^*_2} + \cdots +
		\Delta_{x^*_n}
	\end{equation}
\end{proof}
\begin{theorem}[Относительная погрешность суммы]
	Относительная погрешность суммы приближённых величин (все слагаемые
	одного знака) не превосходит наибольшей относительной погрешности
	слагаемых.
\end{theorem}
\begin{proof}[Доказательство]
	Пусть $x = x_1 + x_2 + \cdots + x_n$ и соответственно $x^* = x^*_1 +
	x^*_2 + \cdots + x^*_n$. Из теоремы~\ref{thm:abs_max_inaccuracy}
	следует:
	$$\Delta_{x^*} \leq \Delta_{x^*_1} + \Delta_{x^*_2} + \cdots +
	\Delta_{x^*_n}$$
	Предположим для определённости, что все $x^*_i > 0$. Тогда, используя
	определение относительной погрешности, имеем:
	$$\delta_{x^*} = \frac{\Delta_{x^*}}{x^*}$$
	Следовательно,
	\begin{equation}
		\delta_{x^*} \leq \frac{\Delta_{x^*_1} + \Delta_{x^*_2} + \cdots
		+ \Delta_{x^*_n}}{x^*_1 + x^*_2 + \cdots + x^*_n}
		\label{eq:2}
	\end{equation}
	Но
	\begin{equation}
		\delta{x^*_i} = \frac{\Delta_{x^*_i}}{x^*_i}, \text{т. е.
		$\Delta_{x^*_i} = \delta_{x^*_i} x^*_i$}.
		\label{eq:3}
	\end{equation}
	Подставляя~\ref{eq:3} в~\ref{eq:2}, получаем:
	\begin{equation}
		\delta{x^*} \leq \frac{\delta_{x^*_1} x^*_1 +
			\delta_{x^*_2} x^*_2 + \cdots + \delta_{x^*_n}
		x^*_n}{x^*_1 + x^*_2 + \cdots + x^*_n}
		\label{eq:4}
	\end{equation}
	Пусть теперь $\overline{\delta} = \max \{\delta_{x^*_1}, \delta{x^*_2},
	\ldots, \delta_{x^*_n}\}$,

	(т.е. $\overline{\delta}$ -- наибольшее из чисел $\delta_{x^*_1},
	\delta_{x^*_2}, \cdots, \delta_{x^*_n})$).

	Тогда неравенство~\ref{eq:4} можно усилить:
	$$\delta_{x^*} \leq \frac{\overline{\delta} x^*_1 +
		\overline{\delta} x^*_2 + \cdots + \overline{\delta}
	x^*_n}{x^*_1 + x^*_2 + \cdots + x^*_n} = \overline{\delta}$$
	Итак,
	\begin{equation}
		\delta_{x^*} \leq \overline{\delta}.
		\label{eq:5}
	\end{equation}
\end{proof}
Совершенно аналогично доказывается теорема в случае, когда все $x^*_i < 0$.
Оценка~\ref{eq:5}, конечно, более грубая, чем оценка относительной погрешности,
даваемая неравенством~\ref{eq:4}.
\subsection{Погрешность разности}
\begin{theorem}[Погрешность разности]
	Пусть $x_1 > x_2 > 0$ и $x^*_1 > x^*_2 > 0$. Образуем разности $x = x_1
	- x_2$ и $x^* = x^*_1 - x^*_2$. Тогда утверждается, что
	\begin{equation}
		\Delta_{x^*} \leq \Delta_{x^*_1} + \Delta_{x^*_2}
		\label{eq:6}
	\end{equation}
	и
	\begin{equation}
		\delta_{x^*} \leq \frac{x^*_1 \delta_{x^*_1} + x^*_2
		\delta_{x^*_2}}{x^*}
		\label{eq:7}
	\end{equation}
\end{theorem}
\begin{proof}[Доказательство]
	Соотношение~\ref{eq:6} для абсолютных погрешностей следует из
	теоремы~\ref{thm:abs_max_inaccuracy}.

	Очевидно, что $x^* > 0$. Поэтому $\delta_{x^*} =
	\frac{\Delta_{x^*}}{x^*}$. Подставляем сюда неравенство~\ref{eq:6}:
	\begin{equation}
		\delta_{x^*} \leq \frac{\Delta_{x^*_1} + \Delta_{x^*_2}}{x^*}
		\label{eq:8}
	\end{equation}
	Но $x^*_1 > 0$ и $^*_2 > 0$. Следовательно, $\Delta_{x^*_1} = x^*_1
	\delta_{x^*_1}$ и $\Delta_{x^*_2} = x^*_2 \delta_{x^*_2}$.
	Таким образом, неравенство~\ref{eq:8} примет окончательный вид:
	\begin{equation}
		\delta_{x^*} \leq \frac{x^*_1 \delta_{x^*_1} + x^*_2
		\delta_{x^*_2}}{x^*}
		\label{7}
	\end{equation}
\end{proof}
\begin{remark}
	Из соотношения~\ref{eq:7} видно, что оценка относительной погрешности
	разности приближённых величин резко ухудшается в том случае, когда
	приближённые величины близки друг к другу. В этом случае знаменатель
	дроби в~\ref{eq:7} становится очень мал, а сама дробь, следовательно,
	весьма велика.
\end{remark}
\begin{example}
	Пусть нам нужно вычислить разность $\sqrt{5.02} - \sqrt{5.01}$ и пусть
	абсолютные погрешности уменьшаемого и вычитаемого не превосходят 0.005.

	Относительная погрешность уменьшаемого тогда не превышает
	$\frac{0.005}{\sqrt{5.02}} \approx 0.0022$,
	и относительная погрешность вычитаемого не превосходит
	$\frac{0.005}{\sqrt{5.01}} \approx 0.0022$.
	В то же время относительная погрешность разности оценивается величиной
	$\frac{0.005 + 0.005}{\sqrt{5.02} - \sqrt{5.01}} \approx 3.2$.
\end{example}

\subsection{Погрешность произведения}
\begin{theorem}[Погрешность произведения]
	Пусть $x = x_1 + x_2$ и $x^* = x^*_1 + x^*_2$. Абсолютная погрешность
	произведения приближённых величин оценивается по формуле:
	\begin{equation}
		\Delta_{x^*} \leq |x^*_2| \Delta_{x^*_1} + |x^*_1|
		\Delta_{x^*_2} + \Delta_{x^*_1} \Delta_{x^*_2},
		\label{eq:9}
	\end{equation}
	а относительная погрешность будет:
	\begin{equation}
		\delta_{x^*} \leq \delta_{x^*_1} + \delta_{x^*_2} +
		\delta_{x^*_3}.
		\label{eq:10}
	\end{equation}
\end{theorem}
\begin{proof}[Доказательство]
	Так как $x = x_1 x_2$ и $x^* = x^*_1 x^*_2$. Будем иметь:
	\begin{equation}
		\begin{split}
			|x - x^*| = |x_1 x_2 - x^*_1 x_2 + x^*_1 x_2
			- x^*_1 x^*_2| \leq\\
			\leq |x_2| |x_1 - x^*_1| + |x^*_1| |x_2 - x^*_2|.
		\end{split}
		\label{eq:11}
	\end{equation}
	Из определения абсолютной погрешности и свойств абсолютных величин
	вытекает, что $|x_2| - |x^*_2| \leq |x_2 - x^*_2| \leq \Delta_{x^*_2}$,
	отсюда
	\begin{equation}
		|x_2| \leq |x^*_2| + \Delta_{x^*_2}.
		\label{eq:12}
	\end{equation}
	Из неравенств~\ref{eq:11} и~\ref{eq:12} получаем:
	$$\Delta_{x^*} \leq
	|x_2| \Delta_{x^*_1} + |x^*_1| \Delta_{x^*_2} \leq
	|x^*_2| \Delta_{x^*_1} + |x^*_1| \Delta_{x^*_2} +
	\Delta{x^*_1} \Delta_{x^*_2}. $$
	Далее, так как $\delta_{x^*} = \frac{\Delta_{x^*}}{|x^*|}$, то
	\begin{equation}
		\begin{split}
			\delta_{x^8} \leq
			\frac{|x^*_2| \Delta_{x^*_1} + |x^*_1| \Delta_{x^*_2} +
			\Delta_{x^*_1} \Delta_{x^*_2}}{|x^*_1 x^*_2|} = \\
			= \frac{|x^*_2| \Delta_{x^*_1}}{|x^*_1 x^*_2|} +
			\frac{|x^*_1| \Delta_{x^8_2}}{|x^*_1 x^*_2|} +
			\frac{\Delta_{x^*_1}\Delta_{x^*_2}}{|x^*_1| |x^*_2|} =
			\frac{\Delta_{x^*_1}}{|x^*_1} +
			\frac{\Delta_{x^*_2}}{|x^*_2|} +\\
			+ \frac{\Delta_{x^*_1}}{|x^*_1|}
			\frac{\Delta_{x^*_2}}{|x^*_2|} =
			\delta_{x^*_1} + \delta_{x^*_2} + \delta_{x^*_1} \delta_{x^*_2}
		\end{split}
	\end{equation}
	Таким образом, соотношение~\ref{eq:10} тоже доказано.
\end{proof}

\subsection{Погрешность частного}
\begin{theorem}[Погрешность частного]
	Пусть $x = \frac{x_1}{x_2}$, $x^* = \frac{x^*_1}{x^8_2}$ и
	$\Delta_{x^*_2} < |x^*_2|$. Тогда абсолютная погрешность частного 2
	приближённых величин оценивается по формуле:
	\begin{equation}
		\Delta_{x^*} \leq \frac{|x^*_1| \Delta_{x^*_2} + |x^*_2|
		\Delta_{x^*_1}}{x^*_2 - |x^*_2| \Delta_{x^*_2}},
		\label{eq:13}
	\end{equation}
	а относительная погршность по формуле:
	\begin{equation}
		\delta_{x^*} \leq \frac{\delta_{x^*_1} + \delta_{x^*_2}}{1 -
		\delta_{x^*_2}}.
		\label{eq:14}
	\end{equation}
\end{theorem}
\begin{proof}[Доказательство]
	Имеем:
	$$|x - x^*| = |\frac{x_1}{x_2} - \frac{x^*_1}{x^*_2}| = \frac{x_1
	x^*_2 - x_2 x^*_1|}{x_2 x^*_2}.$$
	Теперь в числителе прибавим и вычтем величину $x^*_1 x^*_2$:
	\begin{equation}
		\begin{split}
			Delta_{x^*} = |x - x^*| =
			|\frac{x_1 x^*_2 - x^*_1 x^*_2 + x^*_1 x^*_2 - x_2
			x^*_1}{x_2 x^*_2}| \leq \\
			\leq \frac{|x^*_2| |x_1 - x^*_1| + |x^*_1| |x^*_2 -
			x_2|}{|x_2 x^*_2|}.
		\end{split}
	\end{equation}
	Но, как легко видеть, $|x_2| \geq |x^*_2| - \Delta_{x^*_2}$. Поэтому,
	продолжая цепочку неравенств, будем иметь:
	$$\Delta_{x^*} \leq \frac{|x^*_2| \Delta_{x^*_1} + |x^*_1|
	\Delta_{x^*_2}}{|x^*_2 - |x^*_2| \Delta_{x^*_2}|} =
	\frac{|x^*_2| \Delta_{x^*_1} + |x^*_1| \Delta_{x^*_2}}{x^*_2 - |x^*_2|
	\Delta_{x^*_2}}$$
	Далее, деля на $|x^*|$, получим:
	$$\delta_{x^*} = \frac{\Delta_{x^*}}{|x^*|} \leq
	\frac{\Delta_{x^*_2} + |x^*_2| \delta_{x^*_1}}{|x^*_2| -
	\Delta_{x^*_2}}$$
	Разделив числитель и знаменатель дроби на $x^*_2$, получим
	оценку~\ref{eq:14}.
\end{proof}
Мы уже замечали раньше, что точные значения абсолютных (следовательно, и
относительных) погрешностей исходных данных мы обычно не знаем, а знаем лишь
границы, которые эти погрешности не превосходят.

Доказанные нами теоремы об оценках погрешностей арифметических действий
остаются в силе, если в соответствующие формулы подставить не сами абсолютные
погрешности исходных данных, а величины их границ. При этом для применимости
формул~\ref{eq:13} и~\ref{eq:14} необходимо, чтобы граница абсолютной
погрешности для $x_2$ была бы строго меньше $|x^*_2|$.

\chapter{Понятие о функции одной переменной}
\section{Определение функциональной зависимости}
\lettrine[lines=3]{Н}{}
етрудно привести много примеров, когда изменение одних величин влчёт за собой
изменения некоторых других величин, каким-либо образом зависящих от первых.
Так, уровень ртутного столба в термометре тем выше, чем выше температура
окружающей среды. Или длина металличнского стержня тем больше, чем выше его
температура, т. е. уровень ртутного столба в 1-м случае и длина стержня во 2-м
зависят от температуры.

Площадь $S$ треугольника определяется по формуле: $S = \frac{1}{2}ah_a$, где
$h_a$ -- высота, опущенная из вершины треугольника на сторону $a$. Если менять
$h_a$, оставляя $a$ неизменным, то площадь $S$ тоже будет меняться, т.е. можно
сказать, что $S$ зависит от $h_a$.

Подставляя в формулу $y = 2x + 1$ различные значения $x$, будем получать
соответствующие различные значения $y$. Следовательно, и здесь есть
зависимость: $y$ зависит от $x$. Подобных примеров можно приводить сколько
угодно.

\begin{definition}
	Если каждому элементу $x$ из числового множества $X$ поставлено в
	соответствие действительное число $y$, то говорят, что на множестве $X$
	задана функия $y = f(x)$. Множество $X$ называется областью определения
	функции, а множество $Y$ чисел $y$ -- областью значений функции; $x$
	называется независимой переменной или аргументом, а $y$ называется
	функцией этой независимой переменной.
\end{definition}

Запись $y = f(x)$ указывает на тот закон, по которому некоторому значению $x$
ставится в соответствие значение $y$.

Функциональную зависимость можно задавать самыми различными способами.

\subsection{Аналитический способ задания функции}
Чаще всего функциональная зависимость задаётся в виде некоторой комбинации
математических символов (формулы). В этом случае, для того чтобы найти $y$, над
заданным значением $x$ необходимо выполнить все операции, указанные в формуле,
например: $y = 2x + 1$, $y = x^2$, $y = \sqrt{x}$.

Можно определять функцию и с помощью нескольких формул, например:
$$y = \begin{cases}
	x + 1, &\text{если $x > 0$,}\\
	0, &\text{если $x = 0$},\\\
	x - 1, &\text{если $x < 0$}.
\end{cases}$$

Здесь каждому значению $x$ соответствует определённое значение $y$, т. е. мы
имеем функциональную зависимость.

Ещё пример функции: $y = [x]$ -- <<целая часть от $x$>> (ограничимся случаем $x
\geq 0$). Определяется она следующим образом: $y = n$ на полуоткрытом интервале
$[n, n + 1)$ ($n = 0, 1, 2, 3, \ldots$).

Областью определения функциональной зависимости, заданной в виде формулы, мы
будем в дальнейшем считать то множество аргументов, для которого действия,
предписываемые формулой, можно выполнить, и притом однозначно.

Задание функциональной зависимости в виде формулы называется аналитическим
способом задания функции.

\subsubsection{Упражнения}
Найти область определения следующих функций:
\begin{enumerate}
	\item $y = x$
	\item $y = \frac{1}{x}$
	\item $y = \sqrt{x - 1}$
	\item $y = \frac{x}{\sqrt{2x + 1}}$
\end{enumerate}

\subsection{Графическое задание функциональной зависимости}
Теперь опишем геометрический способ задания функциональной зависимости. Для
этого нам придётся вспомнить об изучаемом в элементарной математике понятии
системы координат. Возьмём на плоскости 2 взаимно перпендикулярные прямые --
горизонтальную и вертикальную. На каждой из них зададим положительное
направление. 1-я называется осью абцисс или осью $x$, 2-я -- осью ординат или
осью $y$. Точка $O$ их паересечения называется началом координат. Для измерения
длин отрезков вводим единицу масштаба и откладываем её на осях. Возьмём в
плоскости произвольную точку $M$ и опустим из неё перпендикуляры на оси
координат. Отрезки $ON$ и $OP$ (рис. 2) измерим с помощью выбранной единицы
масштаба. Полученные числа $x$ и $y$ однозначно определяют положение точки $M$
на плоскости, т. е. с введением осей координат и масштаба произвольная точка на
плоскости однозначно определяется парой значений $x$ и $y$, которые и
называются координатами этой точки.

% TODO: чертёж
\chapter[Числовые ряды]{Числовые последовательности и пределы. Числовые ряды}
\chapter[Линейные уравнения]{Линейные алгебраические уравнения и методы их решения}
\chapter{Теория интерполирования}
\chapter{Производная функции одной переменной}
\chapter[Дифференциальное исчисление]{Основные теоремы дифференциального исчисления}
\chapter[Исследование функции]{Исследование функций при помощи производных. Формула Тейлора.
Функциональные ряды}
\chapter{Функции многих переменных}
\chapter[Приближённое рещение уравнений]{Приближённое решение алгебраических и трансцендентных уравнений}
\chapter{Определённый интеграл}
\chapter[Вычисление интегралов]{Приближённое вычисление определённых интегралов}
\chapter{Дифференциальные уравнения}
\section{Приложение. Метод математической индукции}
\end{document}
